\documentclass[12pt]{article}
\usepackage{pgfplots}
\usepackage[margin=1in]{geometry}
\usepackage{listings}
\lstset{
    frame=single,
    breaklines=true,
    basicstyle=\tiny,
}
\title{PS3 Shell}
\author{Kelvin Lin}

\begin{document}

\maketitle

\section{System Properties}
For the purposes of this project, the code ran on a arch-linux machine.

\section{Program Output and Testing}

\begin{lstlisting}
$: 

child process ended with status 139
consumed 0.000000 seconds of user time 
consumed 0.000000 seconds of system time
$: cat PS3.c >output.txt

child process ended with status 0
consumed 0.000000 seconds of user time 
consumed 0.000000 seconds of system time
$: diff PS3.c output.txt

child process ended with status 0
consumed 0.000000 seconds of user time 
consumed 0.000000 seconds of system time
$: cat PS3.c >>output.txt

child process ended with status 0
consumed 0.000000 seconds of user time 
consumed 0.000000 seconds of system time
$: cat PS3.c PS3.c >output2.txt

child process ended with status 0
consumed 0.000000 seconds of user time 
consumed 0.000000 seconds of system time
$: diff output.txt output2.txt

child process ended with status 0
consumed 0.000000 seconds of user time 
consumed 0.000000 seconds of system time
$: cat nonexistent.txt 2>test2.txt

child process ended with status 256
consumed 0.000000 seconds of user time 
consumed 0.000000 seconds of system time
$: cat test2.txt

cat: nonexistent.txt: No such file or directory
child process ended with status 0
consumed 0.000000 seconds of user time 
consumed 0.000000 seconds of system time
$: cat nonexistent.txt 2>>test2.txt

child process ended with status 256
consumed 0.000000 seconds of user time 
consumed 0.000000 seconds of system time
$: cat test2.txt

cat: nonexistent.txt: No such file or directory
cat: nonexistent.txt: No such file or directory
child process ended with status 0
consumed 0.000000 seconds of user time 
consumed 0.000000 seconds of system time
$: cat: nonexistent.txt: No such file or directory

child process ended with status 0
consumed 0.000000 seconds of user time 
consumed 0.000000 seconds of system time

\end{lstlisting}

\section{Source Code}
\subsection{shale}
\begin{lstlisting}
/*
    ECE 460 Computer Operating Systems
    Problem Set 3
    Recursive Interactive Shell

    Author: Kelvin Lin
*/

#include <stdio.h>      // error reporting
#include <errno.h>      // errno
#include <stdlib.h>     // exit
#include <string.h>     // strtok
#include <sys/resource.h> // getrusage()
#include <sys/time.h>
#include <sys/wait.h>   // waitpid
#include <unistd.h>     // exec
#include <sys/types.h>  // open
#include <sys/stat.h>
#include <fcntl.h>

void open_redirect(char *fstream, int io, int flag)
{
    // io = io file descriptor
    // flag = 1 append
    // flag = 0 truncate
    int fd;
    if (!io)
        fd = open(fstream, O_RDONLY);
    else if (flag)
        fd = open(fstream, O_WRONLY|O_CREAT|O_APPEND,0666);
    else
        fd = open(fstream, O_WRONLY|O_CREAT|O_TRUNC,0666);
    if (fd < 0)
    {
        fprintf(stderr,"could not open file %s: %s\n", fstream, strerror(errno));
        exit(0);
    }
    else if (dup2(fd,io) < 0)
    {
        fprintf(stderr,"could not redirect fd %d to fd %d: %s\n",fd, io,strerror(errno));
        exit(0);
    }
    return;
}

int proc_redirect(char *rDirFd)
{
    switch(rDirFd[0])
    {
        case '<':
            printf("lel");
            open_redirect(&rDirFd[1],0,0);
            return 1;
            break;
        case '>':
            if (rDirFd[1] == '>')
                open_redirect(&rDirFd[2],1,1);
            else
                open_redirect(&rDirFd[1],1,0);
            return 1;
            break;
        case '2':
            if (rDirFd[1] == '>')
            {
                if (rDirFd[2] == '>')
                    open_redirect(&rDirFd[3],2,1);
                else
                    open_redirect(&rDirFd[2],2,0);
            }
            return 1;
            break;
        default:
            break;
    }
    return 0;
}

void eggzeck(char *command)
{
    int i = 1;
    char** prog = malloc(256*sizeof(char));
    prog[255] = NULL;
    char delim[10] = " \t\n";
    fcloseall();

    for (prog[0] = strtok(command,delim); (prog[i] = strtok(NULL,delim)) != NULL && i < 255; i++);
    for (i = 1; prog[i] != NULL; i++)
    {
        if (proc_redirect(prog[i]))
            prog[i] = NULL;
    }
        
    execvp(prog[0],prog);
    free(prog);
    exit(0);    // kill the child process
}

int main(int argc, char **argv)
{
    FILE * instream = stdin;
    int printflag = 0;
    if (argc > 1)
    {
        if (!(instream = fopen(argv[1],"r")))
        {
            printf("could not open input file\n");
            exit(0);
        }
        printflag = 1;
    }
    size_t bsize = 256;
    char command[bsize];
    char *cmd = command;
    int pid = 0;
    int status;
    int i = 0;
    struct timeval start, end;
    struct timespec st, endt;
    struct rusage r_usage;
    while(1)    // start shell
    {
        printf("$: ");
        if (fgets(cmd,bsize,instream) == NULL)
            exit(0);
        if (printflag)
            printf("%s\n",command);
        pid = fork();
        if (pid)
        {
            wait3(&status,0,&r_usage);
            printf("child process ended with status %d\n",status);
            printf("consumed %lf seconds of user time \nconsumed %lf seconds of system time\n",
                    (double)(r_usage.ru_utime.tv_sec + (long double)r_usage.ru_utime.tv_usec/1000000),
                    (double)(r_usage.ru_stime.tv_sec + (long double)r_usage.ru_stime.tv_usec/1000000));
            continue;
        }
        eggzeck(command);
    }
    return 0;
}

\end{lstlisting}

\subsection{Test Script}
\begin{lstlisting}

cat PS3.c >output.txt
diff PS3.c output.txt
cat PS3.c >>output.txt
cat PS3.c PS3.c >output2.txt
diff output.txt output2.txt
cat nonexistent.txt 2>test2.txt
cat test2.txt
cat nonexistent.txt 2>>test2.txt
cat test2.txt
cat: nonexistent.txt: No such file or directory

\end{lstlisting}
\subsection{Makefile}

\begin{lstlisting}
all: shale

shale: PS3.c
    gcc PS3.c -o shale

tex: *.tex
    pdflatex PS3.tex

debug: PS3.c
    gcc -g PS3.c -o shale

clean:
    ls -1 | grep -E -v 'PS3.c|Makefile|test.txt|*.tex' | xargs rm -rf
\end{lstlisting}

\end{document}