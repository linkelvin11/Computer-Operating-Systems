\documentclass[12pt]{article}
\usepackage{pgfplots}
\usepackage[margin=1in]{geometry}
\usepackage{listings}
\lstset{
    frame=single,
    breaklines=true,
    basicstyle=\tiny,
}
\title{Grad Option: TCP Flow Control}
\author{Kelvin Lin}

\begin{document}

\maketitle

\section{Source Code}

\subsection{Host}
\lstinputlisting{host.c}

\subsection{Makefile}
\lstinputlisting{Makefile}

\section{Tests}

\subsection{Two Clients}

\subsubsection{host}
\lstinputlisting{two_client_server.txt}

\subsubsection{client 4}
\lstinputlisting{client4.txt}

\subsubsection{client 5}
\lstinputlisting{client5.txt}

\end{document}