\documentclass[12pt]{article}
\usepackage{pgfplots}
\usepackage[margin=1in]{geometry}
\usepackage{listings}
\lstset{
    frame=single,
    breaklines=true,
    basicstyle=\tiny,
}
\title{PS7 Concurrency and Synchronization}
\author{Kelvin Lin}

\begin{document}

\maketitle

\section{Source Code}

\subsection{constants}

\begin{lstlisting}
#include <stdio.h>
#include <stdlib.h>
#include <unistd.h>
#include <sys/mman.h>
#include <signal.h>
#include <errno.h>
#include <string.h>

#define NUMPROC 64
#define NUMLOOP 1e5
#define MYFIFO_BUFSIZ 4096
\end{lstlisting}

\subsection{semaphore}

    \subsubsection{sem.h}
    \begin{lstlisting}
#ifndef SEM_H_
#define SEM_H_

#include "constants.h"

int my_procnum;

struct sem {
    char lock;
    int count;
    int pstatus[NUMPROC];
    int pids[NUMPROC];
};


void sem_init(struct sem *s, int count);

int sem_try(struct sem *s);

void sem_wait(struct sem *s);

void sem_inc(struct sem *s);

#endif
    \end{lstlisting}

    \subsubsection{sem.c}
    \begin{lstlisting}
#include "constants.h"
#include "sem.h"

int tas(volatile char *lock);

void sem_init(struct sem *s, int count) {
    s->count = count;
    s->lock = 0; // unlock

    int i; // zero pstatus and pids
    for (i = 0; i < NUMPROC; i++) {
        s->pstatus[i] = 0;
        s->pids[i] = 0;
    }
}

int sem_try(struct sem *s) {
    int ret;
    while (tas(&s->lock));

    if (s->count > 0){
        s->count--;
        ret = 1;
    }
    else
        ret = 0;
    s->lock = 0;
    return ret;

}

void sigusr1_handler(){
    return; // do nothing; wake from sleep
}

void sem_wait(struct sem *s) {

    s->pids[my_procnum] = getpid();

    while (1) {
        while (tas(&s->lock));

        // block all signals except sigint and sigusr1
        if (signal(SIGUSR1,sigusr1_handler) == SIG_ERR)
            perror("Failed to set signal handler");
        sigset_t mask;
        sigfillset(&mask);
        sigdelset(&mask, SIGUSR1);
        sigdelset(&mask, SIGINT);

        if (s->count > 0) {
            s->count--;
            s->pstatus[my_procnum] = 0;
            s->lock = 0;
            return;
        }
        else {
            s->lock = 0;
            s->pstatus[my_procnum] = 1;
            sigsuspend(&mask);
        }
    }
}

void sem_inc(struct sem *s) {
    while (tas(&s->lock));

    s->count++;

    int i;
    for (i = 0; i < NUMPROC; i++){
        if (s->pstatus[i]){
            kill(s->pids[i], SIGUSR1);
            break;
        }
    }
    s->lock = 0;
}
    \end{lstlisting}

\subsection{fifo}

    \subsubsection{fifo.h}
    \begin{lstlisting}
#include "constants.h"
#include "sem.h"

struct fifo {
    struct sem rd_sem; // read
    struct sem wr_sem; // write
    struct sem ac_sem; // access

    unsigned long buf[MYFIFO_BUFSIZ];
    int head;
    int tail;
};

void fifo_init (struct fifo *f);

void fifo_wr (struct fifo *f, unsigned long d);

unsigned long fifo_rd (struct fifo *f);
    \end{lstlisting}

    \subsubsection{fifo.c}
    \begin{lstlisting}
#include <stdio.h>
#include <stdlib.h>

#include "constants.h"
#include "sem.h"
#include "fifo.h"

void fifo_init (struct fifo *f){

    sem_init(&f->wr_sem,MYFIFO_BUFSIZ);
    sem_init(&f->rd_sem,0);
    sem_init(&f->ac_sem,1);

    f->head = 0;
    f->tail = 0;

    return;
}

void fifo_wr (struct fifo *f, unsigned long d){
    sem_wait(&f->wr_sem);
    sem_wait(&f->ac_sem);

    f->buf[f->head++] = d;
    f->head %= MYFIFO_BUFSIZ;

    sem_inc(&f->rd_sem);
    sem_inc(&f->ac_sem);
}

unsigned long fifo_rd (struct fifo *f){
    unsigned long ret;

    sem_wait(&f->rd_sem);
    sem_wait(&f->ac_sem);

    ret = f->buf[f->tail++];
    f->tail %= MYFIFO_BUFSIZ;

    sem_inc(&f->wr_sem);
    sem_inc(&f->ac_sem);

    return ret;
}
    \end{lstlisting}

\subsection{acid test}

\begin{lstlisting}
#include "constants.h"
#include "sem.h"
#include "fifo.h"

struct fifo *f;
int* child_rem;

void err_exit(char* buf){
    fprintf(stdout,"ERROR: %s: %s\n",buf,strerror(errno));
    exit(1);
}

int err_fork(){
    int frk;
    if ((frk = fork()) == -1)
        err_exit("couldn't fork");

}

void *err_mmap(void *addr, size_t length, int prot, int flags, int fd, off_t offset){
    void *map;
    if ((map = mmap(addr, length, prot, flags, fd, offset)) == MAP_FAILED){
        err_exit("could not map file");
    }
    return map;
}

void kill_children(int parent_id){
    int i;
    for(i = 0; i < NUMPROC; i++) {
        if (i != parent_id && f->ac_sem.pids[i]) {
            kill(f->ac_sem.pids[i], SIGINT);
        }
    }
}

void parent_proc(){
    unsigned long lastread[NUMPROC] = {0};
    unsigned long rd;

    int index;
    // for (index = 0; index < NUMPROC; index++)
    //  lastread[index] = 0;

    // start reading
    int i;
    for(i = 0; i < (NUMPROC-1) * (NUMLOOP); i++){
        rd = fifo_rd(f);
        index = (int)rd/1e6;
        if (lastread[index] > rd){
            kill_children(my_procnum);
            fprintf(stderr,"lastread[%d] = %lu, rd = %lu\n",index,lastread[index],rd);
            fprintf(stderr,"f->head = %d, f->tail = %d\n",f->head,f->tail);
            err_exit("fifo out of order. exiting...");
        }
        lastread[index] = rd;
        // fprintf(stderr,"%08lu\n",rd);
    }
    fprintf(stderr,"done\n");
}

void child_proc(){
    unsigned long my_procid;
    my_procid = my_procnum * 1e6;
    int i;
    for (i = 0; i < NUMLOOP; i++){
        fifo_wr(f,my_procid + i);
    }
    child_rem[0]--;
    if (!child_rem[0]){
        kill(f->ac_sem.pids[0],SIGINT);
        fprintf(stderr,"the deed is done\n");
    }
}

int main(int argc, char** argv){
    unsigned long fifomax;
    unsigned long rd;
    my_procnum = 0;

    child_rem = err_mmap(0,sizeof (int),PROT_READ|PROT_WRITE,MAP_SHARED|MAP_ANONYMOUS,-1,0);
    *child_rem = NUMPROC-1;

    f = err_mmap(0,sizeof (struct fifo),PROT_READ|PROT_WRITE,MAP_SHARED|MAP_ANONYMOUS,-1,0);
    fifo_init(f);

    int i;
    for (i = 1; i < NUMPROC; i++){
        if (err_fork()) {
            
        }
        else {
            my_procnum = i;
            break;
        }
    }
    if (my_procnum)
        child_proc();
    else
        parent_proc();
}
\end{lstlisting}

\subsection{Makefile}

\begin{lstlisting}
all: testprog

testprog: PS7.c sem fifo
    gcc PS7.c tas64.S sem.o fifo.o -o testprog.exe

sem: sem.c
    gcc -c sem.c -o sem.o

fifo: fifo.c
    gcc -c fifo.c -o fifo.o

partA: partA.c sem
    gcc partA.c tas64.S sem.o -o partA.exe

tex: *.tex
    pdflatex PS7.tex

debug: PS5.c
    gcc -g PS5.c -o testprog.exe

clean:
    rm *.o
    rm *.exe
\end{lstlisting}

\section {Test Outputs}

The acid test prints \"the deed is done\" on a successful test.

\subsection {BUFSIZ = $2^{23}$, NUMLOOP = 1e5, \\
64 writers, 1 reader, buffer never fills}
\begin{lstlisting}
[lin4@comm08 PS7_Concurrency_and_Synchronization]$ ./testprog.exe 
the deed is done
\end{lstlisting}

\subsection {BUFSIZ = $2^{12}$, NUMLOOP = 1e5, \\
64 writers, 1 reader, buffer fills many times}
\begin{lstlisting}
[lin4@comm08 PS7_Concurrency_and_Synchronization]$ ./testprog.exe 
the deed is done
\end{lstlisting}

\subsection {remove access semaphore}
\begin{lstlisting}
[lin4@comm08 PS7_Concurrency_and_Synchronization]$ ./testprog.exe 
lastread[1] = 1008026, rd = 1003935
f->head = 3, f->tail = 4
ERROR: fifo out of order. exiting...: Interrupted system call
\end{lstlisting}

\subsection {remove read semaphore}
\begin{lstlisting}
[lin4@comm08 PS7_Concurrency_and_Synchronization]$ ./testprog.exe 
lastread[19] = 19000004, rd = 19000000
f->head = 2286, f->tail = 4
ERROR: fifo out of order. exiting...: Interrupted system call
\end{lstlisting}

\end{document}