\documentclass[12pt]{article}
\usepackage{pgfplots}
\usepackage[margin=1in]{geometry}
\usepackage{listings}
\lstset{
    frame=single,
    breaklines=true,
    basicstyle=\tiny,
}
\title{PS1 System Calls, Error Checking and Reporting}
\author{Kelvin Lin}

\begin{document}

\maketitle

\section{System Properties}
For the purposes of this project, a virtual machine running Ubuntu 14.04 was used.
The virtual machine ran on a windows 7 PC with the following specs:

\begin{tabular}{l l}
CPU: & i7 920 \\
RAM: & 6GB DDR3 1333Mhz \\
\end{tabular}

\section{Program Output and Testing}

\begin{lstlisting}

ls -1 | grep -E -v 'PS1.c|Makefile|test.py|lipsum*|*.tex' | xargs rm -f
gcc PS1.c -o copycat

---test basic functionality---
echo "test input" > testinput.txt
./copycat testinput.txt
test input

---test unspecified buffersize---
./copycat -b
./copycat: option requires an argument -- 'b'

---test invalid buffer sizes---
./copycat -b buffersize -o xout.txt lipsum.txt
ERROR: Invalid buffer size

---test multiple buffers---
./copycat -b 10 -b 50 -o xout.txt lipsum.txt
WARNING: Attempted to initialize two buffersizes

---test unspecified output file---
./copycat -o
./copycat: option requires an argument -- 'o'

---test multiple output files---
./copycat -o xout.txt -o xout2.txt
WARNING: Attempted to open two output files

--test invalid output files---
chmod 000 xout.txt
./copycat -o xout.txt lipsum.txt
ERROR: Could not open output file: Permission denied
chmod 777 xout.txt

---test invalid input files---
./copycat -o xout.txt nonexistent file
ERROR: Unable to open input file -1: No such file or directory

---test multiple file concatenation---
split -n 10 lipsum.txt
./copycat -o xout.txt xa*
diff xout.txt lipsum.txt

---test speed for multiple buffer sizes---
buffersize = 1 bytes
0:14.41 real,	0.49 user,	11.83 sys
buffersize = 2 bytes
0:06.15 real,	0.30 user,	4.99 sys
buffersize = 4 bytes
0:03.29 real,	0.14 user,	2.56 sys
buffersize = 8 bytes
0:02.55 real,	0.08 user,	1.30 sys
buffersize = 16 bytes
0:01.34 real,	0.06 user,	0.63 sys
buffersize = 32 bytes
0:01.09 real,	0.00 user,	0.42 sys
buffersize = 64 bytes
0:00.45 real,	0.02 user,	0.14 sys
buffersize = 128 bytes
0:00.21 real,	0.00 user,	0.08 sys
buffersize = 256 bytes
0:00.10 real,	0.00 user,	0.04 sys
buffersize = 512 bytes
0:00.06 real,	0.00 user,	0.02 sys
buffersize = 1024 bytes
0:00.05 real,	0.00 user,	0.02 sys
buffersize = 2048 bytes
0:00.02 real,	0.00 user,	0.00 sys
buffersize = 4096 bytes
0:00.02 real,	0.00 user,	0.00 sys
buffersize = 8192 bytes
0:00.02 real,	0.00 user,	0.00 sys
buffersize = 16384 bytes
0:00.02 real,	0.00 user,	0.00 sys
buffersize = 32768 bytes
0:00.02 real,	0.00 user,	0.00 sys
buffersize = 65536 bytes
0:00.02 real,	0.00 user,	0.00 sys
buffersize = 131072 bytes
0:00.02 real,	0.00 user,	0.00 sys
buffersize = 262144 bytes
0:00.02 real,	0.00 user,	0.00 sys
ls -1 | grep -E -v 'PS1.c|Makefile|test.py|lipsum*|*.tex' | xargs rm -f
\end{lstlisting}

\section{Buffer Size Analysis}

\begin{tabular}{l l l}
 & Buffer Size (bytes) & Time (s) \\
0 & 1 & 14.41 \\
1 & 2 & 6.15 \\
2 & 4 & 3.29 \\
3 & 8 & 2.55 \\
4 & 16 & 1.34 \\
5 & 32 & 1.09 \\
6 & 64 & 0.45 \\
7 & 128 & 0.21 \\
8 & 256 & 0.10 \\
9 & 512 & 0.06 \\
10 & 1024 & 0.05 \\
11 & 2048 & 0.02 \\
12 & 4096 & 0.02 \\
13 & 8192 & 0.02 \\
14 & 16384 & 0.02 \\
15 & 32768 & 0.02 \\
16 & 65536 & 0.02 \\
17 & 131072 & 0.02 \\
18 & 262144 & 0.02 \\
\end{tabular}
\\
\\
Using the UNIX time command, the copycat program's runtime was recorded for varying buffersizes.
These sizes varied from $2^0$ to $2^18$ bytes long (0 - 256K).
Copycat was tested using a 10mb text file filled with lorem ipsum.
Starting from a small buffer size, the program takes many read/write cycles to move the appropriate bits from one file to another.
The program's rutime is quite large as a result.
However, as the buffer size becomes approximately 2048 bytes long, the amount of time taken by making the system calls becomes negligible compared to the actual reading/writing of the bytes.
As seen from the table, the runtime of copycat settles to around 0.02 seconds for this test case.


\section{Source Code}
\subsection{copycat}
\begin{lstlisting}
/*
	ECE 460 Computer Operating Systems
	Problem Set 1
	System Calls, Error Checking and Reporting

	Author: Kelvin Lin
*/


#include <sys/types.h>
#include <sys/stat.h>
#include <fcntl.h>

#include <unistd.h>

#include <stdio.h>
#include <stdlib.h>
#include <string.h>
#include <errno.h>

#define BUFF_DEFAULT 1024	// default buffer size

int main (int argc, char **argv)
{
	if (argc == 1) return 0;	// if there are no arguments, exit.
	int outfile = 1; 			// default to stdout
	int infiles[100]; 			// default to stdin
	int buffersize = BUFF_DEFAULT;
	int bflag = 0, oflag = 0;	// default option flags to false

	// check for flags
	char opt;
	while((opt = getopt(argc, argv, "b:o:")) != -1)
	{
		switch(opt)
		{
			case 'b': // buffersize specified
				if (bflag)
				{
					fprintf(stderr,"WARNING: Attempted to initialize two buffersizes\n");
				}
				if ((buffersize = atoi(optarg)) == 0)
				{
					fprintf(stderr,"ERROR: Invalid buffer size\n");
					return -1;
				}
				bflag = 1;
				break;
			case 'o': // output file specified
				if (oflag)
					fprintf(stderr,"WARNING: Attempted to open two output files\n");
				if((outfile = open(optarg,O_WRONLY | O_CREAT, 0666)) == -1)
				{
					perror("ERROR: Could not open output file");
					return -1;
				}
				oflag = 1;
				break;
			case ':':
				fprintf(stderr,"ERROR: Option requires an argument: %c\n", (char)optopt);
				return -1;
			default:
				break;
		}
	}

	// initialize buffer to size
	char* buf;
	if ((buf = malloc(buffersize)) == 0)
	{
		fprintf(stderr, 
				"ERROR: Could not allocate %d bytes of memory to buffer with malloc: %s\n", 
				buffersize, strerror(errno));
		return -1;
	}

	// initialize indexing and buffer variables
	int j = optind;
	int offset = j;
	int rd = 0;		// bytes read
	int wr = 0;		// bytes written
	int wlen = 0;	// progress in buffer

	// iterate through non-flag inputs
	for (; j < argc; j++)
	{
		// open the file input from argv
		if (strcmp(argv[j],"-") != 0)
		{
			if ((infiles[j-offset] = open(argv[j],O_RDONLY,0666)) == -1)
			{
				fprintf(stderr,"ERROR: Unable to open input file %d: %s\n",infiles[j-offset],strerror(errno));
				return -1;
			}
		}
		else
			infiles[j-offset] = 0;

		// read from file input and write to ouput
		while((rd = read(infiles[j-offset],buf,buffersize)) != 0)
		{
			if (rd < 0)
			{
				perror("ERROR: Unable to read input");
				free(buf);
				return -1;
			}

			// Check for partial writes & write to file
			wlen = 0;
			do
			{
				wr = write(outfile,buf+wlen,rd-wlen);
				wlen = wlen + wr;
				if (wr < 0)
				{
					perror("ERROR: Unable to write to output");
					free(buf);
					return -1;
				}	
			}
			while(wlen < rd);
		}

		// close the file if its not stdin
		if ((infiles[j-offset] != 0) &&
			(close(infiles[j-offset]) < 0))
		{
			perror("ERROR: There was an error closing an input file");
			free(buf);
			return -1;
		}
	}

	// free buffer memeory
	free(buf);

	// close output file if its not stdout
	if (outfile != 1 &&
		close(outfile) < 0)
	{
		perror("ERROR: There was an error closing the output file");
		return -1;
	}

	// no errors were encountered
	// exit program
	return 0; 
}
\end{lstlisting}

\subsection{Test Script}
\begin{lstlisting}
# Test Script for copycat

import os
import math

def command(com):
	print(com)
	os.system(com)

# generate test files
os.system("make clean")
os.system("make copycat")
os.system("split -n 10 lipsum.txt")

print("\n---test basic functionality---")
command("echo \"test input\" > testinput.txt")
command("./copycat testinput.txt")

print("\n---test unspecified buffersize---")
command("./copycat -b")

print("\n---test invalid buffer sizes---")
command("./copycat -b buffersize -o xout.txt lipsum.txt")

print("\n---test multiple buffers---")
command("./copycat -b 10 -b 50 -o xout.txt lipsum.txt")

print("\n---test unspecified output file---")
command("./copycat -o")

print("\n---test multiple output files---")
command("./copycat -o xout.txt -o xout2.txt")

print("\n--test invalid output files---")
command("chmod 000 xout.txt")
command("./copycat -o xout.txt lipsum.txt")
command("chmod 777 xout.txt")

print("\n---test invalid input files---")
command("./copycat -o xout.txt nonexistent file")

print("\n---test multiple file concatenation---")
command("split -n 10 lipsum.txt")
command("./copycat -o xout.txt xa*")
command("diff xout.txt lipsum.txt")

print("\n---test speed for multiple buffer sizes---")
for i in range(0, 19):
	buffersize = str(pow(2,i))
	print("buffersize = " + str(buffersize) + " bytes")
	os.system("/usr/bin/time -f \"%E real,\t%U user,\t%S sys\" ./copycat -b " + buffersize + " -o xout.txt lipsum2.txt")
os.system("make clean")
\end{lstlisting}
\subsection{Makefile}

\begin{lstlisting}
all: copycat tex

copycat: PS1.c
	gcc PS1.c -o copycat

tex: *.tex
	pdflatex PS1.tex

clean:
	ls -1 | grep -E -v 'PS1.c|Makefile|test.py|lipsum*|*.tex' | xargs rm -f
\end{lstlisting}

\end{document}